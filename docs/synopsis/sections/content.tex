\section{Problem definition}
\label{sec:problem_definition}
The purpose of this bachelor thesis is to implement an interpreter for two reversible languages, where the differences between them are mainly that one is structured and the other unstructered.
Inspiration will be heavily drawn from the languages SRL and RL described in the article "Fundamentals of reversible flowchart languages"[1].
Furthermore we will implement various program transformations, e.g. program inversion, translation between SRL and RL, and RSPT.
To test our implementation and evaluate the practicality of the two languages, we write a collection of test programs of different magnitude.

\section{Boundaries of problem definition}
\label{sec:boundaries_of_problem_definition}

The focus of the project is writing an interpreter for each of the languages RL and SRL in Haskell, that works correctly and has a reasonable operating speed.
We may choose to alter the syntax or semantics of the languages in cooperation with our supervirsor.
The test programs should reflect "real world" problems, in the sense that they should have some level of complexity and purpose.
We will also implement a command-line interface for the interpreters along with a web-based interface for running and testing the languages.

\section{Motivation}
\label{sec:motivation}
Most programming languages today discard information along the execution path, due to the limitation of only being forward deterministic. Reverisble programming languages gives the opportunity for both forward- and backward deterministic execution flows.

\section{Tasks}
\label{sec:tasks}

\begin{enumerate}

  \item We'll have to decide whether to use a parser generator or write our own parser. If we choose to use a parser generator, we'll have to decide on which.

  \item Identify syntax and semantics of core features and full implementation features.

  \item Write the report.

\end{enumerate}

\section{Time Schedule}
\label{sec:time_schedule}

