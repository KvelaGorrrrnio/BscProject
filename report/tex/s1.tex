\section*{Exercise 1}
\label{sec:s1}

%\subsection*{Problem}
%We are given a seqence of propositions. The task is to prove the following:
%\begin{enumerate}
%  \item $a+b = c$
%  \item $a*b = c$
%  \item $a = 7$
%\end{enumerate}
%
%\subsection*{Solution}
%\begin{description}
%  \item[Part (1)] ~\\
%    The task here is to prove the theorem $a^2 + b^2 = c^2$. To do this, we note that the square of a number equals the area of a square with the given sidelength.
%  \item[Task 2] ~\\
%    The task here is to prove the theorem $a^2 + b^2 = c^2$. To do this, we note that the square of a number equals the area of a square with the given sidelength.
%  \item[Task 3] ~\\
%    The task here is to prove the theorem $a^2 + b^2 = c^2$. To do this, we note that the square of a number equals the area of a square with the given sidelength.
%\end{description}


\begin{description}
  \item[Part (1)] ~\\
    The task here is to prove the theorem $a^2 + b^2 = c^2$. To do this, we note that the square of a number equals the area of a square with the given sidelength. \\

    The solution is to view the problem independently.
  \item[Part (2)] ~\\
    The task here is to prove the theorem $a^2 + b^2 = c^2$. To do this, we note that the square of a number equals the area of a square with the given sidelength.
  \item[Part (3)] ~\\
    The task here is to prove the theorem $a^2 + b^2 = c^2$. To do this, we note that the square of a number equals the area of a square with the given sidelength.
\end{description}
