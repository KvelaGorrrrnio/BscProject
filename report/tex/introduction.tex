\chapter*{Introduction}
\addcontentsline{toc}{chapter}{Introduction}%

Many sequential programming languages today are forward deterministic; a given instruction in a program uniquely defines the next state. Most of these languages, however, may discard information along the execution path. Thus, one given instruction in a program may not uniquely define the previous state. These languages are therefore not backward deterministic; they are irreversible.

There are, however, reversible programming languages that do not allow this loss of information. A given instruction still uniquely defines the next state, but it also uniquely defines the previous. Two such languages, proposed in \cite{REV}, are RL and SRL.

\textit{Landauer's Principle}, discussed in \cite{LAN}, states that erased information must be dissipated as heat. That is, the lower limit of power consumption in a microprocessor depends, in some way, on erasure of information. Thus, reversible computation, which does not allow loss of information, can in theory circumvent this lower bound and improve upon the power efficiency of modern computers.

We will in this project, both as proof of concept and for educational purposes, implement an interpreter for each of the proposed languages RL and SRL along with some interesting program transformations, namely program inversion for each of the two languages and program translation \textit{between} the two languages. Furthermore, we will implement --- and demonstrate --- a web based, graphical user interface for the two interpreters.
