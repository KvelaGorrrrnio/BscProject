\section{Usage}

\subsection{Command-Line Interface}

The Command-Line Interface for \texttt{rl} and \texttt{srl} are almost identitical, the only real difference is the naming (rl instead of srl, and vice versa).
Thus only the interface for SRL will be described.
Both Command-Line Interfaces has \texttt{--help} flags, which displays the different modes, options and flags, of which can be chosen and set.

\noindent
Firstly the executable is named \shcmd{$ srl}.
There are 3 modes for the SRL interpreter, as described in Section \ref{sec:implementation_cli}.
Modes are specified as the first argument. The default interface for SRL is \shcmd{$ srl [mode] <file.srl>}.
For a quick recap the modes are:

\begin{description}

  \item[Run]~\\
    Which interprets a SRL program. Run is the default mode, thus can be omitted.
    To interpret the file my-first-script.srl we either call \shcmd{$ srl my-first-script.srl} or \shcmd{$ srl run my-first-script.srl}.

  \item[Translate]~\\
    Which translates a SRL program to RL.
    To translate the file my-first-script.srl we call \shcmd{$ srl translate my-first-script.srl}.

  \item[Invert]~\\
    Which inverts a SRL program.
    To invert the file my-first-script.srl we call\newline\shcmd{$ srl invert my-first-script.srl}.

\end{description}

Instead of using a file as input, a string of SRL code can be given with the c option as \shcmd{$ srl -c "...SRL code..."}. This applies for all modes.
If output is wanted as a json-format, set the -j or {-}{-}json flag.
By default output is directed to stdout. Output can be redirected to a file with th -o option as \shcmd{$ srl input.srl -o output.ext}.

\subsection{Web Interface}

For starting the web server, use \shcmd{$ make server}. This should yield an output ending with
\begin{lstlisting}
  Server has started.
  Web interface is running at http://localhost:3001.
\end{lstlisting}

\noindent
After starting the server and the above-shown output has been printed, every api-call and index page requests to the server is logged to \texttt{stdout}.
The log contains for each request timestamp and the requested path - along with information describing the outcome (erroneous or succesful).
For shutting down the server, use \texttt{Ctrl-C}.
Open the link \url{http://localhost:3001} in a browser, which should bring up the interface described in Section \ref{sec:implementation_web_client}.
