\section*{Usage}
\addcontentsline{toc}{section}{Usage}%

\subsection*{Command-Line Interface}
\addcontentsline{toc}{subsection}{Command-Line Interface}%

 The Command-Line Interface for \texttt{rl} and \texttt{srl} are almost identitical, the only real difference is the naming (rl instead of srl, and vice versa).
 Thus only the interface for \texttt{rl} will be described.
 Both Command-Line Interfaces has \texttt{--help} flags, which displays the different modes, options and flags, of which can be chosen and set.\\

 \todo{Describe modes}\\

\subsection*{Web Interface}
\addcontentsline{toc}{subsection}{Web Interface}%

For starting the web server, use \shcmd{$ make server}. This should yield an output ending with
\begin{lstlisting}
  Server has started.
  Web interface is running at http://localhost:3001.
\end{lstlisting}

After starting the server and the above-shown output has been printed, every api-call and index page requests to the server is logged to \texttt{stdout}.
The log contains for each request timestamp and the requested path - along with information describing the outcome (erroneous or succesful).
For shutting down the server, use \texttt{Ctrl-C}.

\todo{Hosting on two different addresses. (webpack.prod.config.js)}
