\section{Usage}

Here, we will briefly describe how to use the applications that we have implemented.

\subsection{Command-Line Interface}

The command-line interfaces for \texttt{rl} and \texttt{srl} are, in fact, used in exactly the same way. The only difference between using the RL interpreter and the SRL interpreter is the name of the language in which the given program is written. For example, if we want to translate an SRL program \texttt{fib.srl}, we call \texttt{\$ srl translate fib.srl}.% Since RL can be used in the exact same way, only explaining the usage of the SRL should suffice.

The command-line interfaces support, as briefly described in section~\ref{sec:implementation_cli}, three modes. The mode that we want to use must be specified as the first command-line argument. More specifically, we have the modes
\begin{description}

  \item[Run] which simply interprets a program of the chosen language. Run is the default mode and can thus be omitted. For example, to interpret the file \texttt{fib.rl} we either call \texttt{\$ rl fib.rl} or \texttt{\$ rl run fib.rl}.

  \item[Translate] which translates a program from the chosen language to the other. To translate the file \texttt{fib.srl} we call \texttt{\$ srl translate fib.srl}.

  \item[Invert] which inverts a program of the chosen language. To invert the file \texttt{fib.srl} we call \texttt{\$ srl invert fib.srl}.

\end{description}
Instead of using a filename as argument, a string of code can be given verbatim with the `code` option (given as \texttt{-c}), e.g. \texttt{\$ srl -c "[code]"}.  If we want the output formatted as JSON, we can set the `json` flag (given as \texttt{-j} or \texttt{{-}{-}json}).

By default, output is printed to stdout. It can, however, be redirected to a file with the `out` option (given as \texttt{-o}), e.g. \texttt{\$ srl input.srl -o output.ext}.

These options apply for all modes.

When interpreting a program we support one additional flag, namely \texttt{-l} (or \texttt{--log}). This will print a log of the program execution instead of simply the end state of the program. This is used for the step-by-step execution feature end the web interface.

\subsection{Web Interface}

For starting the web server, use \texttt{\$ make server}. This should yield an output ending with
\begin{lstlisting}
  Server has started.
  Web interface is running at http://localhost:3001.
\end{lstlisting}
After starting the server and the above-shown output has been printed, every api-call and index page requests to the server is logged to \texttt{stdout}.
The log contains for each request timestamp and the requested path - along with information describing the outcome (erroneous or succesful).
For shutting down the server, use \texttt{Ctrl-C}.
Open the link \url{http://localhost:3001} in a browser, which should bring up the interface described in Section \ref{sec:implementation_web_client}.
