\section{Testing}


\subsection{Command-Line Interface}

%% Usage
When located at the root of the project, tests for the interpreters/cli can be executed by \shcmd{$ make test}.
For only getting output relevant for testing, build all targets with \shcmd{\$ make} before running the tests.

%% Setup
The test-system is being build and executed by \shcmd{$ stack test}.
The test-system is written in Haskell and is located at \path{/src/test} in Lib.hs and Spec.hs, where Spec.hs is the entry point.
The test-system searches for .rl and .srl files in \path{/src/test/suite}.
For each script, every file with the script filename (including extension) prefixed and .out suffixed are listed.
The script file along with all its .out files is a single test suite.
Outfiles in a test-suite can have the formats described in the table below.

\begin{figure}[H]
  \center
  \begin{tabular}{|l|l|}\hline
    \textbf{Format}             & \textbf{Expected command}\\\hline
    test.srl.out                & \texttt{\$ srl.rl test.srl}\\\hline
    test.srl.log.out            & \texttt{\$ srl.rl test.srl {-}{-}log}\\\hline
    test.srl.invert.out         & \texttt{\$ srl.rl invert test.srl}\\\hline
    test.srl.translate.out      & \texttt{\$ srl.rl translate test.srl}\\\hline
    test.json.srl.out           & \texttt{\$ srl.rl test.srl {-}{-}json}\\\hline
    test.json.srl.log.out       & \texttt{\$ srl.rl test.srl {-}{-}log {-}{-}json}\\\hline
    test.json.srl.invert.out    & \texttt{\$ srl.rl invert test.srl {-}{-}json}\\\hline
    test.json.srl.translate.out & \texttt{\$ srl.rl translate test.srl {-}{-}json}\\\hline
  \end{tabular}
  \caption{Possible outfiles for the file test.srl.}
  \label{fig:test_suite_formats}
\end{figure}

There are currently written 45 tests, some trivial to test correct behavior of specific features, and some less trivial, to test the overall usefulness of the implemented languages.
Two of the less trivial programs are phys and primefac, written both in RL and SRL.

The phys program can be seen found in Appendix \ref{app:phys_srl} and \ref{app:phys_rl}.
The programs simulate a drop from \texttt{y} cm, here set to 16400, with a start velocity \texttt{v} of 0.
In the final state every value is scaled by 10,000.
Here \texttt{y} denotes the distance bias, \texttt{v} is the final velocity and \texttt{s} is the duration of the fall in seconds.
Running any the phys programs yield the following result.

\begin{lstlisting}[language=sh]
S : 10000
s : 57829
v : 56730249
y : -930
\end{lstlisting}

The primefac programs can be found in Appendix \ref{app:primefac_srl} and \ref{app:primefac_rl}.
The programs calculates prime factorials for the given number \texttt{n}.
\texttt{v} is the incrementing counter, to determine the prime factorials and \texttt{p} is the list containing the found numbers.
In the final state \texttt{n} and \texttt{v} should be empty and \texttt{p} a list of prime factorials.
Running any of the primefac programs yield the following result.

\begin{lstlisting}[language=sh]
n : 0
p : [31,59]
v : 0
\end{lstlisting}

\subsection{Web Interface}

%% Client: Packages (Mocha - Chai - Enzyme) - Kun leaf components
The web client is not fully tested, but a test framework is set up and simple tests has been implemented. Components without direct access to application state has been tested. This includes the Button, Radio and Dropdown components.
These files can be found in Appendix \ref{app:test_web_files}.

The framework uses Mocha as its test-framework, Chai as assertion-library and Enzyme for shallow-rendering of components (simulates the rendering of single component-tree).
To run the tests, when standing at the root of the project, run the command \shcmd{$ make test-web}.
This executes tests for both the client and server.

%% Server: Not done yet. Same tool-chain som Client.
For writing server tests, Mocha and Chai has been setup, but no tests has been written.

