\section{Testing}


\subsection{Command-Line Interface}
\todo{CLI testing}\\

%% Usage
When located at the root of the project, tests for the interpreters/cli can be executed by \shcmd{$ make test}.
For only getting output relevant for testing, build all targets with \shcmd{\$ make} before running the tests.\\

%% Setup
The test-system is being build and executed by \shcmd{$ stack test}.
The test-system is written in Haskell and is located at \path{/src/test} in Lib.hs and Spec.hs, where Spec.hs is the entry point.
The test-system searches for .rl and .srl files in \path{/src/test/suite}.
For each script, every file with the script filename (including extension) prefixed and .out suffixed are listed.
The script file along with all its .out files is a single test suite.
Outfiles in a test-suite can have the formats described in Figure \ref{fig:test_suite_formats}.

\begin{figure}[H]
  \center
  \begin{tabular}{|l|l|}\hline
    \textbf{Format}             & \textbf{Expected command output}\\\hline
    test.srl.out                & \texttt{\$ srl.rl test.srl}\\\hline
    test.srl.log.out            & \texttt{\$ srl.rl test.srl {-}{-}log}\\\hline
    test.srl.invert.out         & \texttt{\$ srl.rl invert test.srl}\\\hline
    test.srl.translate.out      & \texttt{\$ srl.rl translate test.srl}\\\hline
    test.json.srl.out           & \texttt{\$ srl.rl test.srl {-}{-}json}\\\hline
    test.json.srl.log.out       & \texttt{\$ srl.rl test.srl {-}{-}log {-}{-}json}\\\hline
    test.json.srl.invert.out    & \texttt{\$ srl.rl invert test.srl {-}{-}json}\\\hline
    test.json.srl.translate.out & \texttt{\$ srl.rl translate test.srl {-}{-}json}\\\hline
  \end{tabular}
  \caption{Possible outfiles for the file test.srl.}
  \label{fig:test_suite_formats}
\end{figure}

%
\noindent
There are currently written 45 tests, some trivial to test correct behavior of specific features, and some less trivial, to test the overall usefulness of the implemented languages.
Two of the less trivial programs are phys.\{rl,srl\} and primefac.\{rl,srl\}.
The phys.\{rl,srl\} can be seen in Figure \ref{fig:phys_files}.
Running the programs yield the result found in Listing \ref{lst:phys_result}.
The programs simulate a drop from \texttt{y} cm, here set to 16400, with a start velocity \texttt{v} of 0.
In the final state every value is scaled by 10,000.
Here \texttt{y} denotes the distance bias, \texttt{v} is the final velocity and \texttt{s} is the duration of the fall in seconds.

\begin{figure}[H]
  \center
  \begin{subfigure}[b]{0.4\textwidth}
    \lstinputlisting[language=srl]{../src/test/suite/phys.srl}
    \caption{phys.srl}
  \end{subfigure}\qquad%
  \begin{subfigure}[b]{0.4\textwidth}
    \lstinputlisting[language=rl]{../src/test/suite/phys.rl}
    \caption{phys.rl}
  \end{subfigure}
  \caption{Program simulating simple physics drop.}
  \label{fig:phys_files}
\end{figure}

\begin{lstlisting}[language=sh,caption={Result for phys.srl},label={lst:phys_result}]
S : 10000
s : 57829
v : 56730249
y : -930
\end{lstlisting}

The primefac programs can be seen in Figure \ref{fig:primefac_files}.
Running the programs yield the result found in Listing \ref{lst:primefac_result}.
The programs calculates prime factorials for the given number \texttt{n}.
\texttt{v} is the incrementing counter, to determine the prime factorials and \texttt{p} is the list containing the found numbers.
In the final state \texttt{n} and \texttt{v} should be empty and \texttt{p} a list of prime factorials.\\

\begin{figure}[H]
  \center
  \begin{subfigure}[b]{0.4\textwidth}
    \lstinputlisting[language=srl]{../src/test/suite/primefac.srl}
    \caption{primefac.srl}
  \end{subfigure}\qquad%
  \begin{subfigure}[b]{0.4\textwidth}
    \lstinputlisting[language=rl]{../src/test/suite/primefac.rl}
    \caption{primefac.rl}
  \end{subfigure}
  \caption{Program simulating prime factorisation.}
  \label{fig:primefac_files}
\end{figure}

\begin{lstlisting}[language=sh,caption={Result for primefac.srl},label={lst:primefac_result}]
n : 0
p : [31,59]
v : 0
\end{lstlisting}

\subsection{Web Interface}

%% Client: Packages (Mocha - Chai - Enzyme) - Kun leaf components
%% Server: Not done yet. Same tool-chain som Client.

\todo{Web client testing}\\
\todo{Web server testing}\\
