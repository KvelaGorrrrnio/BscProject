% Inversion
\section{Program Inversion}

Inverting a program is actually, in our case, very simple. On \cite[p.~104]{REV} we are given a concise `recipe` on how to invert whole programs. We will now go through the implementation of this recipe.

\subsection{Common}

\texttt{Inversion.hs} in the \textit{Common} module contains the inverter that is shared between RL and SRL. The task of this inverter is to invert step operations. As stated earlier, expressions are not invertible, so we can keep them as they are. It is (as seen in \cite[Fig.~20]{REV}) very intuitive: inverting an update with a given operator should result in an update with the operator inverted; inverting a push operation whould result in a pop operation and vice versa; inverting a skip operation should result in a skip. We also have the three added step operations \texttt{swap}, \texttt{init}, and \texttt{free}. We have already gone over these in section~\ref{sec:changes}, but we will reiterate: \texttt{swap} is self-inverse, the inverse of \texttt{init} is \texttt{free} and vice versa. The implementation can be seen in Figure~\ref{fig:commoninvert}.

\begin{figure}[H]
  \lstinputlisting[language=haskell, firstline=5, lastline=19]{../src/src/Common/Inversion.hs}
  \caption{The step operation inverter.}\label{fig:commoninvert}
\end{figure}

\subsection{RL}

The RL-specific inverter can be found in \texttt{Inversion.hs} under the \textit{RL} module and is, of course, based on the inverter described in \cite[Fig.~19]{REV}:

If we want to invert an RL program --- that is, a list of blocks --- we want to reverse the list of blocks and invert each individual block.

To invert a block, we first want to invert the come-from assertion and use the result as the jump. Next, we will reverse the list of step operations and invert each of these individually. Finally, we will invert the jump and use the result as the come-from assertion.

\begin{figure}[H]
  \lstinputlisting[language=haskell, firstline=9, lastline=26]{../src/src/RL/Inversion.hs}
  \caption{The step operation inverter.}\label{fig:commoninvert}
\end{figure}

\subsection{SRL}
