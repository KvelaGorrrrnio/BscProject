% Inversion
\section{Program Inversion}

Inverting a program is actually, in our case, very simple. On \cite[p.~104]{REV} we are given a concise inverter for each of the languages along with the shared core. We will now go through the implementation of these inverters.

\subsection{Common}
\label{sec:invcommon}

\texttt{Inversion.hs} in the \textit{Common} module contains the inverter that is shared between RL and SRL. The task of this inverter is to invert step operations. As stated earlier, expressions are not invertible, so we can keep them as they are. It is (as seen in \cite[Fig.~20]{REV}) very intuitive: inverting an update with a given operator should result in an update with the operator inverted; inverting a push operation whould result in a pop operation and vice versa; inverting a skip operation should result in a skip. We also have the three added step operations \texttt{swap}, \texttt{init}, and \texttt{free}. We have already gone over these in section~\ref{sec:changes}, but we will reiterate: \texttt{swap} is self-inverse, the inverse of \texttt{init} is \texttt{free} and vice versa. The implementation can be seen in Figure~\ref{fig:commoninvert}.

\begin{figure}[H]
  \lstinputlisting[language=haskell, firstline=5, lastline=19]{../src/src/Common/Inversion.hs}
  \caption{The step operation inverter.}\label{fig:commoninvert}
\end{figure}

\subsection{Inverting RL Programs}

The RL-specific inverter can be found in \texttt{Inversion.hs} under the \textit{RL} module and is, of course, based on the inverter described in \cite[Fig.~19]{REV}.\\

\noindent If we want to invert an RL program --- that is, a list of blocks --- we want to reverse the list of blocks and invert each individual block.\\

\noindent To invert a block, we first want to invert the come-from assertion and use the result as the jump. Next, we will reverse the list of step operations and invert each of these individually. Finally, we will invert the jump and use the result as the come-from assertion.\\

\noindent The inversion of come-from assertions and jumps is rather simple. A \texttt{goto} becomes a \texttt{from}, an \texttt{if} becomes a \texttt{fi}, an exit becomes an entry and vice versa.\\

\noindent The implementation can be seen in Figure~\ref{fig:rlinvert}.

%\noindent \todo{More stuff on inversion.}

\begin{figure}[H]
  \lstinputlisting[language=haskell, firstline=9, lastline=26]{../src/src/RL/Inversion.hs}
  \caption{The RL program inverter.}\label{fig:rlinvert}
\end{figure}

\subsection{Inverting SRL Programs}

The SRL-specific inverter can be found in \texttt{Inversion.hs} under the \textit{SRL} module. This is based on the inverter presented in \cite[Fig.~18]{REV}.\\

\noindent Inverting an SRL program corresponds to inverting a block.\\

\noindent If the block is a step operation, we simply invert the step operation as described in section~\ref{sec:invcommon}.\\

\noindent If the block is a conditional, we swap the test and the assertion and recursively invert the two blocks inside.\\

\noindent Similarly, if the block is a loop, we swap the test and the assertion and invert the two blocks inside.\\

\noindent Finally, if the block is a sequence of two blocks, we invert each and reverse the order.\\

\noindent The implementation can be seen in Figure~\ref{fig:srlinvert}.

\begin{figure}[H]
  \lstinputlisting[language=haskell, firstline=7, lastline=11]{../src/src/SRL/Inversion.hs}
  \caption{The RL program inverter.}\label{fig:srlinvert}
\end{figure}
