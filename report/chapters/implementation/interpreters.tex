
%% The interpreters
\section{The Interpreters}

This section describes the implementation of the different parts of the interpreters; the common core and the RL- and SRL specifics.

\subsection{Changes}\label{sec:changes}

It is important to note that we have made a few changes to the languages. We have modified some of the syntax described in Fig.~\ref{fig:rl_srl_syntax_and_structure}, but also added some new step operations and core features. We have not, however, altered the very nature of any of the languages.

We found the paper~\cite{REV} to be a bit unclear regarding types. The languages are described on the more hypothetical level, and the paper seems to assume some kind of type inference. We tried implementing a static check similar to type inference, but we deemed it too unrobust. This brings us to one of our most fundamental additions: explicit variable declarations. Each program must start with zero or more variable declarations with a specified type. For example, \texttt{list int a} declares variable \texttt{a} to be of type \texttt{list int}, etc. Declaring a variable multiple times, or using a variable in the program that has not been declared, throws a runtime error.

Also, while the languages as described in the paper have lists limited to one dimension, our implementation of the core feature set supports indexing in an arbitrary number of dimensions. That is, you could declare a list variable as \texttt{list list int l} and index on this list with i.e. \texttt{l[2,5]} (assuming that the indices are not out of bounds). This, in particular, makes it a non-trivial task to implement robust type inference.

We have added three new step operations to the languages: \texttt{swap}, \texttt{init}, and \texttt{free}. \texttt{swap} has already been described in \cite[p.~99]{REV}, however, it has not been included in the formal syntax of the languages. It is a simple, but powerful, step operation that is imperative for the \textit{Fibonacci-pair function} shown as an unstructured flowchart on \cite[p.~99]{REV} and as a structured flowchart on \cite[p.~93]{REV}. \texttt{swap} works by simply swapping the two variable operands and is, thus, its own inverse. Trying to swap two variables of different types will throw a runtime error.

\texttt{init} was defined and implemented by us as a way to quickly obtain a list of zero-values of arbitrary dimensions. That is, $\texttt{init} \ l \ [x,y,z]$ will initialise variable $l$ to be an $x \times y \times z$ table containing only zeroes. This is, in particular, useful in the translation from RL to SRL; Here, we must begin the resulting SRL program by initialising an $(n+1) \times (n+1) \times 3$ table, where $n$ is the number of blocks in the source RL program. This would, if we did not have \texttt{init} at our disposal, require three nested loops and three temporary variables which are never used again. One strict rule to ensure reversibility is that the variable being initialised \textit{must} be an empty list before the execution. If this is not the case, a runtime error is thrown.

The inverse to \texttt{init} is \texttt{free}, which has a very similar syntax; $\texttt{free} \ l \ [x,y,z]$ for some variable $l$ and expressions $x$, $y$, and $z$ will, if the dimensions and lengths of the list match $[x,y,z]$, and if all values in $l$ are zero-cleared, reset $l$ to an empty list. Otherwise a runtime error will be thrown. This guarantees the reversibility of \texttt{init} and \texttt{free}. The inverse to \texttt{free} is, naturally, \texttt{init}.

We do not consider the addition of \texttt{init} and \texttt{free} to be \textit{too} radical; after all, the same functionality could be simulated by simply having a sufficiently long sequence of push- (if \texttt{init}) or pop (if \texttt{free}) operations.

We have also added two update operators: multiplication and division. Here we must be cautious: The right-hand side of a multiplication update must not be zero. This would lead to loss of information and make the program irreversible. Also, the right-hand side of a division update must be a divisor of the left-hand side --- a potential division rest would be lost, and the program would therefore not be reversible. If these rules are not respected, a runtime error is thrown. We have that $x \timeseq e$ is an inverse to $x \diveq e$ and vice versa.

Because expressions do not influence reversibility, one can add any arbitrary operator to the languages. We found a few common operators to be a nice addition to the languages --- some of them are even mentioned in the paper, but not explicitly included in the formal syntax. More specifically, we have added
\begin{description}
\item[Binary operators]~
  \begin{itemize}
  \item Power
  \item Equality
  \item Comparison
  \item Modulo
  \item Logical conjunction
  \item Logical disjunction
\end{itemize}
\item[Unary operators]~
\begin{itemize}
  \item Negation
  \item The sign function
  \item Logical negation
  \item Size of list
  \item A predicate to determine if a variable contains only zero-values
\end{itemize}
\end{description}
The addition of arbitrary indexing applies to expressions as well; we can now express e.g. $\texttt{(top } e_1 \texttt{)[}e_2\texttt{]}$, for some expressions $e_1$ and $e_2$, without a problem by first evaluating $\texttt{(top }e_1\texttt{)}$ and then indexing on the resulting value with $e_2$.

Finally, some of the RL-specific syntax has undergone some minor changes. The rather verbose syntax of the conditional come-from assertion has been changed from $\texttt{fi} \ e \ \texttt{from} \ l_1 \ \texttt{else} \ l_2$ to simply $\texttt{fi} \ e \ l_1 \ l_2$. Similarly, the syntax of the conditional jump has been changed from $\texttt{if} \ e \ \texttt{goto} \ l_1 \ \texttt{else} \ l_2$ to simply $\texttt{if} \ e \ l_1 \ l_2$.

%% Core of the interpreter
\subsection{Interpreter Core}

\subsubsection{Abstract syntax tree and utilities}
\texttt{AST.hs} under the \textit{Common} module (Appendix~\ref{app:Common_AST_hs}) contains the parts of the specific abstract syntax trees that are shared between the two languages. Furthermore, it contains some extra data types and functions that are used in both the \textit{RL}- and \textit{SRL} modules.

At the most fundamental level are the identifiers, the values, and their types. An identifiers marks a variable in a step operation and can contain zero or more indices. Having no indices corresponds, naturally, to just a variable $x$, while having one or more indices corresponds to indexing on a variable $x$. The implementation is as follows:
\lstinputlisting[language=haskell, firstnumber=21, firstline=21, lastline=21]{../src/src/Common/AST.hs}
Values in the languages are, as previously stated, either integers or lists. The implementation looks as follows:
\lstinputlisting[language=haskell, firstnumber=9, firstline=9, lastline=9]{../src/src/Common/AST.hs}
Note that the list value has an extra field that specifies its type -- that is, essentially, the depth of the list. More specifically, it is a recursive data type such that
\lstinputlisting[language=haskell, firstnumber=172, firstline=172, lastline=172]{../src/src/Common/AST.hs}
This makes it trivial to perform dynamic type checks in the interpreter.

The store of our interpreters --- that is, the variable table --- is represented as a hashmap with strings as keys and \texttt{Value}s as values:
\lstinputlisting[language=haskell, firstnumber=30, firstline=30, lastline=30]{../src/src/Common/AST.hs}
Thus, reading a variable in an expression amounts to simply doing a lookup on the hashmap, while an update simply has to modify it.

Next, we have expressions. These are exactly as described in the paper, but with the additions covered in section~\ref{sec:changes}. The definition is as follows:
\lstinputlisting[language=haskell, firstnumber=74, firstline=74, lastline=81]{../src/src/Common/AST.hs}
Just as described, \texttt{Exp} is a recursive data type terminated by a literal value or a variable lookup. Since we do not support list literals, a literal value is assumed to be an integer.


Other than that, we have the \texttt{Binary} construct which takes a binary operator (\texttt{BinOp}) and two expressions, recursively, as arguments. Binary operators can easily be defined separately. The same goes for the \texttt{Unary} construct; this takes a unary operator (\texttt{UnOp}) and an expression, recursively. Unary operators, as binary operators, can easily be defined separately. The \texttt{Index} construct is simply for indexing on subexpressions. The purpose of the \texttt{Parens} construct is to mark parentheses occurring in the original program; this makes it trivial to write a program to a file which, when parsed, has the guaranteed same expression precedences, and therefore the same behaviour, as the original.

Finally, we have step operations. Revisiting~\ref{fig:common}, we can focus on the definition of the step operations. We have that a step operation $a$ is given by
\[
\begin{matrix*}[l]
  {a} & ::= & {x}\mathrel{\oplus}= e \\
             &  |  & {x}[ e]\mathrel{\oplus}= e \\
             &  |  & \texttt{push }{x}\ {x} \\
             &  |  & \texttt{pop  }{x}\ {x} \\
             &  |  & \texttt{skip},
\end{matrix*}
\]
where $x$ is an arbitrary variable name. If we add our additional step operations --- \texttt{swap}, \texttt{init}, and \texttt{free} --- and features --- i.e. arbitrary indexing --- as described in section~\ref{sec:changes}, we have
\[
\begin{matrix*}[l]
  {a} & ::= & {x}\iota\mathrel{\oplus}= e       \quad\quad\quad  & \iota  & ::= & \epsilon \\
      &  |  & \texttt{push }{x}\iota \mathbin{} {x}\iota                   &        & |   & \iota' \\
      &  |  & \texttt{pop } \ \ {x}\iota \mathbin{} {x}\iota \\
      &  |  & \texttt{skip }                                      & \iota' & ::= & [e] \\
      &  |  & \texttt{swap } x\iota \mathbin{} x\iota                    &        & |   & [e]\iota' \\
      &  |  & \texttt{init } x \mathbin{} \iota' \\
      &  |  & \texttt{free } x \mathbin{} \iota'
\end{matrix*}
\]
The implementation of this was straightforward:
\lstinputlisting[language=haskell, firstnumber=41, firstline=41, lastline=48]{../src/src/Common/AST.hs}

\subsubsection{Errors and log}
The \textit{Common} module also contains the implementation of the possible program errors in \texttt{Error.hs} (Appendix~\ref{app:Common_Error_hs}) and the log functionality in \texttt{Log.hs} (Appendix~\ref{app:Common_Log_hs}), but we won't go into too much detail as to \textit{how} they are implemented. To understand the interpreters, however, we must know which errors can be potentially thrown:
\begin{description}
  \item[\texttt{RuntimeError}] covers the errors thrown at runtime. This includes type errors, division by zero, etc.
  \item[\texttt{ParseError}] is simply a wrapper for potential parse errors. This is to be able to handle all errors in a single place.
  \item[\texttt{StaticError}] covers the errors that can be identified in a static check. While this is in the \textit{Common} module, static errors are mostly exclusive to RL. This includes errors thrown because of duplicate entries, duplicate exits, etc. An important thing to note is that it throws an error if the first block of an RL program is not the entry point or if the last block is not an exit point. This makes the translator from RL to SRL more similar to the one described in the paper.
  \item[\texttt{Custom}] is a generic error type used for anything that we did not feel was covered by any of the previous types. Namely, if no file (or code, if the \texttt{-c} has been set --- more on that later) has been given to the interpreter in the command-line interface, a `custom` error is thrown.
\end{description}
Having specialised error types makes it easier to handle the concrete messages, all in one place, using the \texttt{Show} typeclass.

The log is simply a list of specialised messages that can either be a step operation or an error. When a program is interpreted, each step operation that is executed, along with the state of the program after the execution, is recorded in the log. If an error is thrown, this is recorded as well before the interpretation halts. This gives us a simple debugger that will work with both the command-line interface and the web-based user interface.

\subsubsection{Interpretation}
\texttt{Interp.hs} under the \textit{Common} module (Appendix~\ref{app:Common_Interp_hs}) contains the core interpreting engine --- that is, the execution of step operations and evaluation of expressions. We define the core monad transformer stack to contain the state monad (for the store), the except monad (for error handling), and the writer monad (for the log). The \texttt{exec} function (lines 80-225), is the one that executes a step operation and updates the program state accordingly.

It would consume too much space to describe the complete implementation of \texttt{exec}, so an example or two should suffice. Lines 83-106 show the implementation of the variable update; we first check if the identifier being updated occurs in the expression on the right-hand side. If it does, we throw a runtime error. Next, we fetch the value of the variable being updated and then evaluate the right-hand expression. If one of these is not an integer, we throw an error. Before actually updating, we check if the update is either a multiplication update or division update, and if so, if they respect the rules mentioned in section~\ref{sec:changes}. Finally, we update the store with the new value. This implicitly handles indexing; \texttt{rd}, occurring on line 87, reads a variable from the store and, if one or more indices are provided, digs deeper into the list to return the corresponding entry. \texttt{adjust}, seen on line 106, works in the same way, but replaces each list on the path and performs a given operation on the innermost index of the list.

Lines 153-162 show the implementation of the swap operation; we first fetch the value of the two variables being swapped, then check if they are of the same type. If not, we throw a runtime error. Otherwise, we update the first operand to the value of the second and vice versa.

Lines 236-303 show how the evaluation of expressions has been implemented. Again, two concrete examples should suffice. On lines 242-253 we handle the case of the binary operators denoting addition, subtraction, bitwise exclusive or, power, multiplication, equality, and inequality, since these can be handled in the same manner. We first evaluate the left and right operands and check their types. Except for the case of equality, only integer values are supported. Thus, if the operator is not `equals` or `not equals`, the two subexpressions \textit{have} to be evaluated to integers. If this rule is not respected, a runtime error is thrown. If nothing fails, the operator is applied and the resulting value is returned.

On lines 274-279 we handle the case where an arithmetic negation or the sign function occurs. The approach here is very similar to that of the binary operator evaluation, however, having only one operand to handle, we can do in a more concise manner. We evaluate the operand and make sure that it results in an integer. If it doesn't, we throw a runtime error. Otherwise, we apply the operator and return the resulting value.

Do note that we have implemented short-circuit evaluation for logical conjunction and logical disjunction (lines 265-272). That is, if the left-hand side of a logical conjunction evaluates to false, we do not need to evaluate the right-hand side. If the left-hand side of a logical disjunction evaluates to true, we do not need to evaluate the right-hand side. This is demonstrated in \texttt{shortcircuit.rl} (Appendix~\ref{app:shortcircuit_rl}).

%Note that \texttt{logError} (for example seen on line 3 of \ref{fig:exupdate}) is simply a function that records a given error in the log before throwing it and halting the execution.

\subsubsection{Parser}
The parser was implemented using Parsec along with the Token helper module to perform a lexical analysis implicitly. Not surprisingly, the common parser module defines the parsers of the common language structures. These will be used in the specialised parsers.





%% RL specific interpreter
\subsection{RL Interpreter}

\subsubsection{Abstract syntax tree}

In section~\ref{sec:rlandsrl} we considered the syntax of RL; a program must consist of one or more blocks, where a block is defined by its label and has a come-from assertion, zero or more step operations, and a jump. With the step operations already implemented, the only thing to worry about is how to represent our program. \texttt{AST.hs} under the \textit{RL} module (Appendix~\ref{app:RL_AST_hs}) contains the complete representation. As one will notice, we decided to define the abstract syntax tree as an association list with labels as keys and blocks as values:
\lstinputlisting[language=haskell, firstnumber=14, firstline=14, lastline=14]{../src/src/RL/AST.hs}
Note that \texttt{Label} is simply a type synonym for \texttt{String}.

A block, then, is defined as a three-tuple consisting of a come-from assertion, a list of step operations, and a jump. Thus, we have
\lstinputlisting[language=haskell, firstnumber=18, firstline=18, lastline=18]{../src/src/RL/AST.hs}
It should come to no surprise that \texttt{From} and \texttt{Jump} are the data types that represent come-from assertions and unstructured jumps, respectively. Their implementations, practically copied from the paper, are as follows:
\begin{figure}[H]
  \centering
\begin{subfigure}{0.45\textwidth}
  \lstinputlisting[language=haskell, firstnumber=27, firstline=27, lastline=30]{../src/src/RL/AST.hs}
\end{subfigure}
\qquad
\begin{subfigure}{0.45\textwidth}
  \lstinputlisting[language=haskell, firstnumber=36, firstline=36, lastline=39]{../src/src/RL/AST.hs}
\end{subfigure}
\end{figure}

\subsubsection{Interpretation}
The interpreter for RL can be found in \texttt{Interp.hs} under the \textit{RL} module (Appendix~\ref{app:RL_Interp_hs}). It is pretty simple and essentially a small wrapper around the core feature set from the \textit{Common} module.

First thing is that we have extended our core monad transformer stack to also contain the reader monad. This is in order to read from our abstract syntrax tree --- that is, our association list of blocks --- from a constant resource without having to throw it around in every recursive call.

The \texttt{interp} function (lines 31-57), which interprets a single block and passes control flow to the next, works by receiving two arguments: the label of the block that we are coming from and the label of the block we are jumping to. We assume that an empty label as first operand means that we are at the entry of the program. When calling the function, we simply look up the label of the block we are jumping to in the abstract syntax tree --- from the static check we assume that the corresponding block exists. Then, to interpret the block, we simply make sure that we have gone down the right path (that is, our come-from assertion passes the test regarding the come-from label). Then we execute the list of step operations and call \texttt{interp} recursively --- now with the current label as come-from label and with the label dictated by the unstructured jump as the label we are jumping to.

Now, we have briefly mentioned the static check of the RL interpreter, but we have not gone over what exactly it does. \texttt{Static.hs} under the \textit{RL} module (Appendix~\ref{app:RL_Static_hs}) checks that a few rules regarding the form of an RL program are respected --- each of which can be established statically. More specifically, we check for the following:
\begin{itemize}
  \item The program starts with an entry block
  \item The program ends with an exit block
  \item The program does not contain two or more entries
  \item The program does not contain two or more exits
  \item All labels that are somehow referred to are indeed assigned to a block
  \item The program does not contain a label that is assigned twice or more
\end{itemize}
Thus, the static check simply makes sure that our program is well formed and that each jump and come-from assertion is well defined, so to speak.

\subsubsection{Parser}
When parsing, we simply parse for a list of one or more \texttt{(Label,Block)} pairs. Each pair is parsed by identifying a block label followed by a colon, a come-from assertion, a list of zero or more step operations as defined in the common parser, and a jump.

%% SRL specific interpreter
\subsection{SRL Interpreter}

\subsubsection{Abstract syntax tree}

As with RL, we have already looked at the syntax of SRL in section~\ref{sec:rlandsrl}. To summarise: A program consists of exactly one main program block. This block can either be a step operation, a conditional ($\texttt{if} \ e_1 \ \texttt{then} \ b_1 \ \texttt{else} \ b_2 \ \texttt{fi} \ e_2$), a loop ($\texttt{from} \ e_1 \ \texttt{do} \ b_1 \ \texttt{loop} \ b_2 \ \texttt{until} \ e_2$), or recursively a sequence of two blocks ($b_1 \ b_2$). Fortunately, this translates nicely into an implementation in Haskell which is found in \texttt{AST.hs} under the \textit{SRL} module (Appendix~\ref{app:SRL_AST_hs}). For the block, we have defined the \texttt{Block} data type as follows:
\lstinputlisting[language=haskell, firstnumber=17, firstline=17, lastline=20]{../src/src/SRL/AST.hs}
Since a program is represented as a single block, we have defined the abstract syntax tree accordingly. More specifically, we have that
\lstinputlisting[language=haskell, firstnumber=13, firstline=13, lastline=13]{../src/src/SRL/AST.hs}

\subsubsection{Interpretation}

The process of interpreting the abstract syntax tree --- that is, a single block --- is fairly simple. \texttt{Interp.hs} under the \textit{SRL} module (Appendix~\ref{app:SRL_Interp_hs}) contains the \texttt{interp} function (lines 24-51) which takes a block as argument and interprets it. The function covers the different types of blocks:
\begin{itemize}
  \item If it is a step operation, we simply execute it.

  \item If it is a conditional, we determine which block to execute depending on the test. After executing the block, we evaluate the assertion and compare it to the evaluation of the test. If they match, we are done. If not, the program breaks the rules of reversibility and we have to throw a runtime error.

  \item If the block is a loop, the assertion must hold at the entry of the loop. We then execute the first block and evaluate the test. If the test evaluates to true, we are done. If not, we execute the second block and make sure that the assertion now does not hold. If it still holds, the program breaks the rules of reversibility and we have to throw a runtime error. Otherwise, we repeat.

  \item Finally, if it is a sequence of two blocks, we recursively interpret the first block and the the second block, as described, and in that order.
\end{itemize}
%All this is done in the \texttt{interp} function in \texttt{Interp.hs} of the \textit{SRL} module and can be seen in Appendix~\ref{app:srlinterp}.

%\begin{figure}[H]
%  \lstinputlisting[language=haskell, firstline=24, lastline=51]{../src/src/SRL/Interp.hs}
%  \caption{The \texttt{interp} function that interprets a block and thus a program.}\label{fig:srlinterp}
%\end{figure}

\subsubsection{Parser}

This parser simply parses a block as one of the block types defined for SRL. When parsing more than one block, we convert the list to a sequence by folding the list of blocks.
