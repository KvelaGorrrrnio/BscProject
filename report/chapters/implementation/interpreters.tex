
%% The interpreters
\section{The Interpreters}

\subsection{Changes}\label{sec:changes}

It is important to note that we have made a few changes to the languages. We have modified some of the syntax described in Figure~\ref{fig:rl_srl_syntax_and_structure}, but also added some new step operations and core features. We have not, however, altered the very nature of any of the languages.\\

\noindent We found the paper\cite{REV} to be a bit unclear regarding types. The languages are described on the more hypothetical level, and the paper seems to assume some kind of type inference. We tried implementing a static check similar to type inference, but we deemed it too unrobust. This brings us to one of our most fundamental additions: explicit variable declarations. Each program must start with zero or more variable declarations with a specified type. For example, \texttt{list int a} declares variable \texttt{a} to be of type \texttt{list int}, etc. Using a variable in the program that has not been declared throws a runtime error.

Also, while the languages as described in the paper have lists limited to one dimension, our implementation of the core feature set supports indexing in an arbitrary number of dimensions. That is, you could declare a list variable as \texttt{list list int l} and index on this list with i.e. \texttt{l[2,5]} (assuming that the indices are not out of bounds). This, in particular, makes it a non-trivial task to implement robust type inference.\\

%Furthermore, in order to make the languages more user friendly we have tried to remove the concept of overflow. That is, instead of representing integers

\noindent We have added three new step operations to the languages: \texttt{swap}, \texttt{init}, and \texttt{free}. \texttt{swap} has already been described in \cite[p.~99]{REV}, however, it was not included in the formal syntax of the languages. It is a simple, but powerful, step operation that is imperative for the \textit{Fibonacci-pair function} shown as an unstructured flowchart on \cite[p.~99]{REV} and as a structured flowchart on \cite[p.~93]{REV}. \texttt{swap} works simply by swapping the two variable operands and is, thus, its own inverse. Trying to swap two variables of different types will throw a runtime error.

\texttt{init} was defined and implemented by us as a way to quickly obtain a list of zero-values in an arbitrary number of dimensions. That is, $\texttt{init} \ l \ [x,y,z]$ will initialise list $l$ to be a 3-dimensional table of lengths $x$, $y$, and $z$ consisting of zeroes. This is, in particular, useful in the translation from RL to SRL; Here, we must begin the resulting SRL program by initialising an $n \times n \times 3$ table, where $n$ is the number of blocks in the source RL program. This would, if we did not have \texttt{init} at our disposal, require three nested loops and three temporary variables which are never used again.

The inverse to \texttt{init} is \texttt{free}, which has a very similar syntax; $\texttt{free} \ l \ [x,y,z]$ will, if the dimensions and lengths of the list match $[x,y,z]$, reset list $l$ to an empty list. Do they not match, however, a runtime error will be thrown. This guarantees the reversibility of \texttt{init} and \texttt{free}. The inverse to \texttt{free} is, naturally, \texttt{init}.

We do not consider the addition of \texttt{init} and \texttt{free} to be \textit{too} radical; after all, the same functionality could be simulated by simply having a sufficiently long sequence of \texttt{push}- (if \texttt{init}) or \texttt{pop} (if \texttt{free}) operations.\\

\noindent We have also added two update operators: multiplication and division. Here we must be cautious: The right-hand side of a multiplication update must not be zero. This would lead to loss of information and make the program irreversible. Also, the right-hand side of a division update must be a divisor of the left-hand side --- a potential division rest would be lost, and the program would therefore not be reversible. If these rules are not respected, a runtime error is thrown. We have that $x \timeseq e$ is an inverse to $x \diveq e$ and vice versa. \\

\noindent Because expressions do not influence reversibility, one can add any arbitray operator to the languages. We found a few known operators to be a nice addition to the languages --- some of them are even mentioned in the paper, but not included in the formal syntax. We will keep this description brief:
\begin{description}
\item[Binary operators]~
  \begin{itemize}
  \item Power (infix, \texttt{**})
  \item Equality (infix, \texttt{=} and \texttt{!=})
  \item Relational (infix, \texttt{<}, \texttt{<=}, \texttt{>}, \texttt{>=})
  \item Modulo (infix, \texttt{\%})
  \item Logical conjunction (infix, \texttt{\&\&} or \texttt{and})
  \item Logical disjunction (infix, \texttt{||} or \texttt{or})
\end{itemize}
\item[Unary operators]~
\begin{itemize}
  \item Negation (prefix, \texttt{-} or \texttt{neg})
  \item Sign (prefix, \texttt{\~} or \texttt{sig})
  \item Logical negation (prefix, \texttt{!} or \texttt{not})
  \item Size of list (prefix, \texttt{\#} or \texttt{size})
  \item A predicate to determine if a variable contains only zero-values (prefix, \texttt{null})
\end{itemize}
\end{description}~

\noindent Finally, some of the RL-specific syntax has undergone some minor changes. The rather verbose syntax of the conditional come-from assertion has been changed from
\[\texttt{fi} \ e \ \texttt{from} \ l \ \texttt{else} \ l\]
\noindent to simply
\[\texttt{fi} \ e \ l \ l.\]
\noindent Similarly, the syntax of the conditional jump has been changed from
\[\texttt{if} \ e \ \texttt{goto} \ l \ \texttt{else} \ l\]
\noindent to simply
\[\texttt{if} \ e \ l \ l.\]

%% Core of the interpreter
\subsection{Interpreter Core}

The core features shared between the two languages are contained in the \textit{Common} module.

\subsubsection{Abstract syntax tree and utilities}
\texttt{Common/AST.hs} contains the parts of the specific abstract syntax trees that are shared between the two languages. Furthermore, it contains some extra data types and functions that are used in both the RL- and SRL modules.

At the most fundamental level are the identifiers, the values, and their types. Identifiers mark a variable in the program and can contain zero or more indices. No indices correspond, naturally, to just a variable $x$, while one or more indices correspond to indexing on a list $x[e,\dots]$. The implementation can be seen in~\ref{fig:identifier}.

Values in the languages are, as previously stated, either integers or lists. The implementation of \texttt{Value} can be seen in Figure~\ref{fig:val}. Note that list values have an extra field that specifies their type -- that is, essentially, the depth of the list; this makes it trivial to perform dynamic type checks in the interpreter. The implementation of \texttt{Type} as a recursive data type can be seen in~\ref{fig:type}.

\begin{figure}[H]
  \lstinputlisting[language=haskell, firstline=21, lastline=21]{../src/src/Common/AST.hs}
  \caption{Implementation of identifiers}\label{fig:identifier}
\end{figure}
\begin{figure}[H]
  \lstinputlisting[language=haskell, firstline=9, lastline=9]{../src/src/Common/AST.hs}
  \caption{Implementation of the fundamental value types}\label{fig:val}
\end{figure}
\begin{figure}[H]
  \lstinputlisting[language=haskell, firstline=174, lastline=174]{../src/src/Common/AST.hs}
  \caption{Implementation of the types}\label{fig:type}
\end{figure}

\noindent The store of our interpreters --- that is, the variable state --- is represented by the type synonym \texttt{VarTab} as a hashmap with strings as keys and \texttt{Value}s as values. Thus, reading a variable in an expression amounts to simply do a lookup on the hashmap, while an update simply has to, well, update the store.\\

\noindent Next, we have expressions. These are exactly as described in the paper, but with the additions covered in \ref{sec:changes}. The implementation can be seen in Figure~\ref{fig:exp}. As expected, \texttt{Exp} is a recursive data type terminated by a literal value or a variable lookup. Since we do not support list literals, a literal value is assumed to be an integer.

\begin{figure}[h]
  \lstinputlisting[language=haskell, firstline=79, lastline=85]{../src/src/Common/AST.hs}
  \caption{Implementation of expressions}\label{fig:exp}
\end{figure}

\noindent Other than that, we have the \texttt{Binary} structure which takes a binary operator (\texttt{BinOp}) and two expressions, recursively, as arguments. Binary operators can easily be defined separately. The same goes for the \texttt{Unary} structure; this takes a unary operator (\texttt{UnOp}) and an expression, recursively. Unary operators, as binary operators, can easily be defined separately. The final expression is simply to mark parentheses; this makes it trivial to write a program to a file which, when parsed, has the guaranteed same expression precedences, and thus, same behaviour as the original.\\

\noindent Finally, we have step operations. Revisiting~\ref{fig:common}, we can focus on the definition of the step operations. We have that a step operation $a$ is given by
\[
\begin{matrix*}[l]
  {a} & ::= & {x}\mathrel{\oplus}= e \\
             &  |  & {x}[ e]\mathrel{\oplus}= e \\
             &  |  & \texttt{push }{x}\ {x} \\
             &  |  & \texttt{pop  }{x}\ {x} \\
             &  |  & \texttt{skip}
\end{matrix*}
\]
If we add our additional step operations and features (as described in~\ref{sec:changes}), we have
\[
\begin{matrix*}[l]
  {a} & ::= & {x}([ e,\dots])\mathrel{\oplus}= e \\
      &  |  & \texttt{push }{x}([e,\dots])\ {x}([e,\dots]) \\
      &  |  & \texttt{pop  }{x}([e,\dots])\ {x}([e,\dots]) \\
      &  |  & \texttt{skip} \\
%  \dots \\
      &  |  & \texttt{swap} \ x([e,\dots]) \ x([e,\dots]) \\
      &  |  & \texttt{init} \ x \ [e,\dots] \\
      &  |  & \texttt{free} \ x \ [e,\dots],
\end{matrix*}
\]
where $([e,\dots])$ means that you can optionally index in an arbitrary number of dimensions. The implementation of this was straightforward, and the corresponding code can be seen in Figure~\ref{fig:step}.

\begin{figure}[H]
  \lstinputlisting[language=haskell, firstline=46, lastline=53]{../src/src/Common/AST.hs}
  \caption{Implementation of the step operations}\label{fig:step}
\end{figure}

\subsubsection{Errors and log}
The \textit{Common} module also contains the implementation of the error- and log types, but we won't go into too much detail as to \textit{how} they are implemented. To understand the interpreters, however, a few things about these must be clear.\\

\noindent Each error type is represented in a data structure \texttt{Error} in \texttt{Common/Error.hs}:
\begin{description}
  \item[\texttt{RuntimeError}] covers the errors thrown at runtime. This includes type errors, division by zero, etc.
  \item[\texttt{ParseError}] is simply a wrapper for potential parse errors. This is to be able to handle all errors in a single place.
  \item[\texttt{StaticError}] covers the errors that can be identified in a static check. While this is in the \textit{Common} module, static errors are exclusive to RL. This includes errors thrown because of duplicate entries, duplicate exits, etc.
  \item[\texttt{Custom}] is a generic error type used for anything that we did not feel was covered by any of the previous types. Namely, if no file (or code, if the \texttt{-c} has been set --- more on that later) has been given to the interpreter in the command-line interface, a `custom` error is thrown.
\end{description}

\noindent Having specialised error types makes it easier to handle the concrete messages, all in one place, using the \texttt{Show} typeclass.\\

\noindent The log (found in \texttt{Common/Log.hs}) is simply a list of specialised messages (our own \texttt{Message} type) that can either be a step operation or an error. When a program is interpreted, each step operation that is executed, along with the state of the program after the execution, is recorded in the log. If an error is thrown, this is recorded as well before the interpretation terminates. This gives us a simple debugger that will work with both the command-line interface and the web user interface.

\subsubsection{Interpretation}
\texttt{Common/Interp.hs} contains the core interpreting engine --- that is, the execution of step operations and evaluation of expressions. We define the core monad transformer, with type synonym \texttt{VarState}, to contain the state monad (for the store), the except monad (for error handling), and the writer monad (for the log). The concrete \texttt{VarState} can be seen in Figure~\ref{fig:varstate}.

\begin{figure}[H]
  \lstinputlisting[language=haskell, firstline=24, lastline=24]{../src/src/Common/Interp.hs}
  \caption{The core monad transformer}\label{fig:varstate}
\end{figure}

\noindent The \texttt{exec} function, whose type signature can be seen in Figure~\ref{fig:exec}, is the function that executes a step operation and updates the program state (\texttt{VarTab}) accordingly.

%% Type signature of exec
\begin{figure}[H]
  \lstinputlisting[language=haskell, firstline=78, lastline=78]{../src/src/Common/Interp.hs}
  \caption{Type signature of \texttt{exec}}\label{fig:exec}
\end{figure}

\noindent It would consume too much space to show the complete implementation of \texttt{exec}, so an example or two (as seen in Figure~\ref{fig:examples}) should be enough. Figure~\ref{fig:exupdate} shows the implementation of the variable update; we first check if the identifier being updated occurs in the expression on the right-hand side. If it does, we throw a runtime error. Next, we fetch the value of the variable being updated and then evaluate the right-hand expression. If one of these is not an integer, we throw an error. Before actually updating, we check if the update is either a multiplication update or division update, and if so, if they respect the rules mentioned in \ref{sec:changes}. Finally, we update the store with the new value. This implicitly handles indexing; \texttt{rd}, seen on line 5 of Figure~\ref{fig:exupdate}, reads a variable from the store and, if one or more indices are provided, digs deeper into the list to return the corresponding entry. \texttt{adjust}, seen on the last line of the same example, works in the same way, but replaces each list on the path and performs a given operation on the innermost index of the list.

Figure~\ref{fig:exswap} shows the implementation of the swap operation; we first fetch the value of the two variables being swapped, then check if they are of the same type. If not, we throw a runtime error. Otherwise, we update the first operand to the value of the second and vice versa.

%% The two examples of exec
\begin{figure}[h]

  \begin{subfigure}{\textwidth}
    \lstinputlisting[language=haskell, firstline=81, lastline=104]{../src/src/Common/Interp.hs}
    \caption{Implementation of the variable update.}\label{fig:exupdate}
  \end{subfigure}

  \begin{subfigure}{\textwidth}
    \lstinputlisting[language=haskell, firstline=167, lastline=176]{../src/src/Common/Interp.hs}
    \caption{Implementation of the swap operation.}\label{fig:exswap}
  \end{subfigure}

  \caption{Two examples of step operation execution.}\label{fig:examples}

\end{figure}

\noindent Evaluation of expressions takes place in the \texttt{eval} function. The type signature can be seen in Figure~\ref{fig:eval}.

%% Type signature of eval
\begin{figure}[H]
  \lstinputlisting[language=haskell, firstline=250, lastline=250]{../src/src/Common/Interp.hs}
  \caption{Type signature of \texttt{eval}}\label{fig:eval}
\end{figure}

\noindent Again, two concrete examples (as seen in Figure~\ref{fig:evalexamples}) should suffice. In Figure~\ref{fig:exbinary} we handle the case of the binary operators denoting addition, subtraction, bitwise exclusive or, power, multiplication, equality, and inequality. We first evaluate the left and right operands and check their types. Except for the case of equality, only integer values are supported. Thus, if the operator is not `equals` or `not equals`, the two operands \textit{have} to be integers. If this rule is not respected, a runtime error is thrown. If nothing fails, the operator is applied and the resulting value is returned.

  In Figure~\ref{fig:exunary} we handle the case where an arithmetic negation or the sign function occurs. The approach here is very similar to that in Figure~\ref{fig:exbinary}, however, having only one operand to handle, we can do in a more concise manner. We evaluate the operand and make sure that it results in an integer. If i doesn't, we throw a runtime error. Otherwise, we apply the operator and return the resulting value.

%% The two examples of eval
\begin{figure}[h]

  \begin{subfigure}{\textwidth}
    \lstinputlisting[language=haskell, firstline=256, lastline=267]{../src/src/Common/Interp.hs}
    \caption{Partial implementation of binary operator expression evaluation.}\label{fig:exbinary}
  \end{subfigure}

  \begin{subfigure}{\textwidth}
    \lstinputlisting[language=haskell, firstline=288, lastline=293]{../src/src/Common/Interp.hs}
    \caption{Partial implementation of unary operator expression evaluation.}\label{fig:exunary}
  \end{subfigure}

  \caption{Two examples of expression evaluation.}\label{fig:evalexamples}

\end{figure}

\noindent Note that \texttt{logError} (for example seen on line 3 of \ref{fig:exupdate}) is simply a function that records a given error in the log before throwing it and halting the execution.

\subsubsection{Parser}
The parser was implemented using Parsec along with the Token helper module to perform a lexical analysis implicitly. Not surprisingly, the common parser module defines the parsers of the common language structures. These will be use in the specialised parsers.





%% RL specific interpreter
\subsection{RL Interpreter}

The RL-specifics regarding the interpreter are contained in the \textit{RL} module.

\subsubsection{Abstract syntax tree}

In section~\ref{sec:rlandsrl} we considered the syntax of RL; a program must consist of one or more blocks, where a block is defined by its label and has a come-from assertion, zero or more step operations and a jump. With the step operations already implemented, the only thing to worry about is how to structure the abstract syntax tree. We decided to represent the abstract syntax tree as an association list with labels as keys and blocks as values (Figure~\ref{fig:rlast}). A block, then, is defined as a three-tuple consisting of a come-from assertion, a list of step operations, and a jump (Figure~\ref{fig:rlblock}). Note that \texttt{Label} is simply a type synonym for \texttt{String}.

 \begin{figure}[h]
  \lstinputlisting[language=haskell, firstline=20, lastline=20]{../src/src/RL/AST.hs}
  \caption{The representation of an RL block.}\label{fig:rlblock}
\end{figure}

\begin{figure}[h]
  \lstinputlisting[language=haskell, firstline=16, lastline=16]{../src/src/RL/AST.hs}
  \caption{The abstract syntax tree for RL.}\label{fig:rlast}
\end{figure}

\noindent It should come to no surprise that \texttt{From} and \texttt{Jump} are the data types that represent come-from assertions and unstructured jumps, respectively. Their implementations, practically copied from the paper, can be seen in Figure~\ref{fig:from} and Figure~\ref{fig:jump}.

\begin{figure}[h]
  \lstinputlisting[language=haskell, firstline=29, lastline=32]{../src/src/RL/AST.hs}
  \caption{The representation of come-from assertions.}\label{fig:from}
\end{figure}

\begin{figure}[H]
  \lstinputlisting[language=haskell, firstline=38, lastline=41]{../src/src/RL/AST.hs}
  \caption{The representation of unstructured jumps.}\label{fig:jump}
\end{figure}

\subsubsection{Interpretation}
The interpreter for RL can be found in \texttt{RL/Interp.hs}. It is pretty simple and essentially a small wrapper around the core feature set from the Common module.

First thing is that we have extended our monad transformer to also contain the reader monad. This is in order to read from our abstract syntrax tree --- that is, our association list of blocks --- from a constant resource without having to throw it around in every recursive call. The new monad transformer has \texttt{ProgState} as type synonym and its definition can be seen in Figure~\ref{fig:progstate}.

\begin{figure}[h]
  \lstinputlisting[language=haskell, firstline=14, lastline=14]{../src/src/RL/Interp.hs}
  \caption{The extended monad transformer \texttt{ProgState}.}\label{fig:progstate}
\end{figure}

\noindent The \texttt{interp} function, which interprets a single block and passes control flow to the next, works by receiving two arguments: the label of the block that we currently in and the label of the block we are jumping to. We assume that an empty label as first operand means that we are at the entry of the program. When interpreting a block, we simply make sure that we have gone down the right path (that is, our come-from assertion passes the test). Then we execute the list of step operations and jump to the next block as dictated by the unstructured jump associated with our block. The \texttt{interp} function can be seen in Figure~\ref{fig:interp}.

\begin{figure}[H]
  \lstinputlisting[language=haskell, firstline=33, lastline=67]{../src/src/RL/Interp.hs}
  \caption{The \texttt{interp} function.}\label{fig:interp}
\end{figure}

%% SRL specific interpreter
\subsection{SRL Interpreter}

The SRL-specific interpreter implementation can be found in the \textit{SRL} module.

\subsubsection{Abstract syntax tree}

As with RL, we have already looked at the syntax of SRL in section~\ref{sec:rlandsrl}. To summarise:
We managed to translate this directly:


\subsubsection{Interpretation}
