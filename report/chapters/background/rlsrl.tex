
% går så videre til det mere specifikke med RL og SRL
% og snakker om den formelle syntaks
\section*{RL and SRL}
\addcontentsline{toc}{section}{RL and SRL}%
\noindent Reversible Language and Structured Reversible Language --- RL and SRL for short, repectively --- are two reversible flowchart languages proposed in~\cite{REV}. RL is an assembly-like language with unstructured jumps, while SRL, as the name indicates, is a similar language but with control structures and no jumps. The two languages share a common subset of features --- namely their expressions and step operations. The grammar of the two languages along with the common structures are given in~\cite{REV} and can be seen in Figure~\ref{fig:rl_srl_grammar_and_structure}.\\

\noindent As we see, RL, described in Figure~\ref{fig:rlspec}, consists of one or more (RL-)blocks, where each block is (uniquely) defined by a label and has a come-from assertion, zero or more step operations and an unstructured jump to any block in the program. Note that a well-formed RL program must contain exactly one entry and one exit.

SRL is more linear; as we see in Figure~\ref{fig:srlspec}, a program consists of exactly one (SRL-)block, where a block may either be an atomic step operation, a control structure --- that is, an \texttt{if}- or a \texttt{do-until} structure --- or recursively a sequence of two blocks. Sequences of blocks are executed in order, and thus, an SRL program looks a lot like a program written in common programming languages such as Python or C, to name a few. Note that a block in RL is different from a block in SRL.\\

\noindent As mentioned, the two languages share expressions and step operations along with the basic value types. A brief overview:\\

\begin{description}
  \item[Values and types]~\
  \begin{itemize}
    \item A variable can either be mapped to an integer or to a list of integers.
    \item We only have integer literals; to obtain a certain list of integers, one has to bind each of the wanted values to a variable and push it to a list.
    \item Boolean values are represented by integers. A non-zero integer value will evaluate to true, and a zero integer value will evaluate to false.
  \end{itemize}
  \item [Step operations]~\
  \begin{itemize}
    \item To modify a variable, we have the variable update. The variable update only supports reversible operators, namely addition and subtraction. Indexing is supported in a single dimension. One rule is that the identifier being updated must not occur in the right-hand expression since such an update is irreversible.
    \item You can push the value of a variable onto a list with the \texttt{push} step operation. The variable being pushed must be zero-cleared afterwards.
    \item You can pop the top value from a list into a variable with the \texttt{pop} step operation. The variable must be zero-cleared beforehand.
    \item The \texttt{skip} step operation does nothing.
  \end{itemize}
  \item [Expressions]~\
  \begin{itemize}
    \item We can have any arbitrary binary operator between two expressions, recursively.
    \item An expression terminates with an integer literal or a variable lookup.
    \item The \texttt{top} expression is unary and returns the top of the operand.
    \item The \texttt{empty} predicate is unary and returns true if the operand is an empty list and false otherwise.
  \end{itemize}
\end{description}

% figur der viser grammatikken her tak
\begin{figure}[]

  \begin{subfigure}{\textwidth}
    \center
    $$\begin{matrix*}[l]
      {p} & ::= & {b} & \quad & {b} & ::= & a             & | & \texttt{ if }e\texttt{ then }b\texttt{ else }b\texttt{ fi }e\\
                 &     &           &        &            &  |  & b\ b & | & \texttt{ from }e\texttt{ do }b\texttt{ loop }b\texttt{ until }e\\
    \end{matrix*}$$
    \caption{Structured reversible language \textit{SRL}.}
    \label{fig:srlspec}
  \end{subfigure}

  \begin{subfigure}{\textwidth}
    \center
    $$\begin{matrix*}[l]
      {q} & ::= & {d}^+                                      & & {k} & ::= & \texttt{from }l & & {j} & ::= & \texttt{goto }l\\
      {d} & ::= & {l}:\ {k}\ {a}^*\ {j} &       &            &  |  & \texttt{fi }e\texttt{ from }l\texttt{ else }l &
        & & | & \texttt{if }e\texttt{ goto }l\texttt{ else }l\\
    \end{matrix*}$$
    \caption{Unstructured reversible language \textit{RL}.}
    \label{fig:rlspec}
  \end{subfigure}

  \begin{subfigure}{\textwidth}
    \center
    $$\begin{matrix*}[l]
      {a} & ::= & {x}\oplus= e & \quad\quad\quad &  e & ::= & {c} \ | \ {x} \ | \ {x}[ e] \ | \  e\otimes e \ | \ \texttt{top }{x} \ | \ \texttt{empty }{x}\\
                 &  |  & {x}[ e]\oplus= e &     & {c} & ::= & 0\ | \ 1\ |\ \cdots\ |\ 4294967295\\
                 &  |  & \texttt{push }{x}\ {x}      &     & \otimes    & ::= & \oplus \ | \ * \ | \ / \ | \ \cdots\\
                 &  |  & \texttt{pop  }{x}\ {x}      &     & \oplus     & ::= & +  \ | \  -  \ | \  \textasciicircum\\
                 &  |  & \texttt{skip}\\
    \end{matrix*}$$
    \caption{Reversible step operations and expressions.}
  \end{subfigure}

  \begin{subfigure}{\textwidth}
    \center
    $$\begin{matrix*}[l]
      \mathtt{SRL}:     & {p}\in\mathtt{SRL}  & & {b}\in\mathtt{Blk}\\
      \mathtt{RL}:      & {q}\in\mathtt{RL}   & & {d}\in\mathtt{RLBlk} & & {j}\in\mathtt{Jump}  & & {k}\in\mathtt{From} & & {l}\in\mathtt{Label}\\
      \mathtt{SRL, RL}: & {a}\in\mathtt{Step} & &  e\in\mathtt{Exp}   & & {c}\in\mathtt{Const} & & {x}\in\mathtt{Var}  & & \oplus,\otimes\in\mathtt{Op}\\
    \end{matrix*}$$
    \caption{Syntax domains of \textit{SRL} and \textit{RL}.}
  \end{subfigure}

  \caption{Syntax of the two reversible flowchart languages.}
  \label{fig:rl_srl_grammar_and_structure}

\end{figure}
