\section{Reversible Flowcharts}
Flowcharts are a natural way to model imperative programs. Similarly, \textit{reversible} imperative programs can be naturally modeled by reversible flowcharts. The two models are similar --- however, to actually guarantee reversibility in reversible flowcharts, we may define some additional attributes. More specifically, we may want any join point between two edges in a flowchart to be acknowledged through an assertion which can determine the origin of the control flow. The definition~\cite[p.~89]{REV} states that \\

\say{A \textit{reversible flowchart $F$} is a finite directed graph with three types of nodes, each of which represents an \textit{atomic operation} [...]} \\

\noindent T. Yokoyama et al.\ move on to describe the three atomic operations:
\begin{description}
  \item[(a)] the \textit{step} operation performs some injective (i.e.\ reversible) operation and passes the control flow to the outgoing edge,
  \item[(b)] the \textit{test} passes the control flow to one of two outgoing edges depending on a predicate $e$, and
  \item[(c)] the \textit{assertion} joins two edges and passes the control flow depending on a predicate $e$.
\end{description}
\noindent The atomic operations (as illustrated in \cite[Fig.~2]{REV}) can be seen (reproduced) in Fig.~\ref{fig:atop}. Having these as elementary building blocks of our reversible flowcharts, it is trivial to invert a program; one simply has to invert each edge and, naturally, any step operation in the flowchart. This, in turn, means that the inversion of a test is an assertion and vice versa. The inversion of each atomic operation (as illustrated in \cite[Fig.~3]{REV}) can be seen in Fig.~\ref{fig:atopinv}.

This outlines the basic theory behind reversible flowcharts. We can put this theory into practice and yield a \textit{reversible flowchart language} --- an incarnation of reversible flowcharts. Two such languages are RL and SRL. %is the general idea of reversible flowcharts, and two incarnations of this model are RL and SRL.
\begin{figure}
  \centering
  \begin{subfigure}[t]{0.3\textwidth}
    \centering
    \scalebox{0.8}{%
      \begin{tikzpicture}[auto, thick, node distance=2cm, >=triangle 45]
        \node (a) [stmt] {$a$};
        \node [output] (up)    [above of=a, node distance=1.1cm]{};
        \node [input]  (end)   [right of=a, node distance=1.5cm]{};
        \node [output] (start) [left  of=a, node distance=1.5cm]{};
        \node [output] (down)  [below of=a, node distance=1.1cm]{};

        \path[->] (start) edge node {} (a);
        \path[->] (a) edge node {} (end) ;

        \path[draw=none] (up) node {} (a);
        \path[draw=none] (down) node {} (a);
      \end{tikzpicture}
    }
    \caption{Step}
  \end{subfigure}
  \begin{subfigure}[t]{0.3\textwidth}
    \centering
    \scalebox{0.8}{%
      \begin{tikzpicture}[auto, thick, node distance=2cm, >=triangle 45]
        \node (a) [test] {$e$};
        \node [input]  (start) [left of=a, node distance=1.5cm]{};
        \node [output]  (up)   [above of=a, node distance=1.2cm]{};
        \node [output]  (ur)   [right of=up, node distance=1.2cm]{};
        \node [output] (down)  [below of=a, node distance=1.2cm]{};
        \node [output] (dr)    [right of=down, node distance=1.2cm]{};

        \path[->] (start) edge node {} (a);
        \path[-] (a) edge node {$t$} (up);
        \path[->] (up) edge node {} (ur);
        \path[-] (a) edge node {$f$} (down);
        \path[->] (down) edge node {} (dr);
      \end{tikzpicture}
    }
    \caption{Test}
  \end{subfigure}
  \begin{subfigure}[t]{0.3\textwidth}
    \centering
    \scalebox{0.8}{%
      \begin{tikzpicture}[auto, thick, node distance=2cm, >=triangle 45]
        \node [assert] (a)    {$e$};
        \node [input]  (up)   [below of=a, node distance=1.2cm]{};
        \node [input]  (dr)   [left of=up, node distance=1.2cm]{};
        \node [input]  (down) [above of=a, node distance=1.2cm]{};
        \node [input]  (ur)   [left of=down, node distance=1.2cm]{};
        \node [output] (end)  [right of=a, node distance=1.5cm]{};

        \path[-] (dr) edge node {} (up);
        \path[->] (up) edge node {$f$} (a);
        \path[-] (ur) edge node {} (down);
        \path[->] (down) edge node {$t$} (a);
        \path[->] (a) edge node {} (end);
      \end{tikzpicture}
    }
    \caption{Assertion}
  \end{subfigure}
  \caption{The three atomic operations of reversible flowcharts}\label{fig:atop}
\end{figure}

\begin{figure}
  \centering
  \begin{subfigure}[t]{0.3\textwidth}
    \centering
    \scalebox{0.8}{%
      \begin{tikzpicture}[auto, thick, node distance=2cm, >=triangle 45]
        \node (a) [stmt] {$a^{-1}$};
        \node [output] (up)    [above of=a, node distance=1.3cm]{};
        \node [input]  (end)   [right of=a, node distance=1.5cm]{};
        \node [output] (start) [left  of=a, node distance=1.5cm]{};
        \node [output] (down)  [below of=a, node distance=1.3cm]{};

        \path[->] (a) edge node {} (start);
        \path[->] (end) edge node {} (a) ;

        \path[draw=none] (up) node {} (a);
        \path[draw=none] (down) node {} (a);
      \end{tikzpicture}
    }
    \caption{Step}
  \end{subfigure}
  \begin{subfigure}[t]{0.3\textwidth}
    \centering
    \scalebox{0.8}{%
      \begin{tikzpicture}[auto, thick, node distance=2cm, >=triangle 45]
        \node [assert] (a)    {$e$};
        \node [input]  (up)   [below of=a, node distance=1.2cm]{};
        \node [input]  (dr)   [right of=up, node distance=1.2cm]{};
        \node [input]  (down) [above of=a, node distance=1.2cm]{};
        \node [input]  (ur)   [right of=down, node distance=1.2cm]{};
        \node [output] (end)  [left of=a, node distance=1.5cm]{};

        \path[-] (dr) edge node {} (up);
        \path[->] (up) edge node {$f$} (a);
        \path[-] (ur) edge node {} (down);
        \path[->] (down) edge node {$t$} (a);
        \path[->] (a) edge node {} (end);
      \end{tikzpicture}
    }
    \caption{Assertion}
  \end{subfigure}
  \begin{subfigure}[t]{0.3\textwidth}
    \centering
    \scalebox{0.8}{%
      \begin{tikzpicture}[auto, thick, node distance=2cm, >=triangle 45]
        \node (a) [test] {$e$};
        \node [input]  (start) [right of=a, node distance=1.5cm]{};
        \node [output]  (up)   [above of=a, node distance=1.2cm]{};
        \node [output]  (ur)   [left of=up, node distance=1.2cm]{};
        \node [output] (down)  [below of=a, node distance=1.2cm]{};
        \node [output] (dr)    [left of=down, node distance=1.2cm]{};

        \path[->] (start) edge node {} (a);
        \path[-] (a) edge node {$t$} (up);
        \path[->] (up) edge node {} (ur);
        \path[-] (a) edge node {$f$} (down);
        \path[->] (down) edge node {} (dr);
      \end{tikzpicture}
    }
    \caption{Test}
  \end{subfigure}
  \caption{Inversion of the three atomic operations of reversible flowcharts}\label{fig:atopinv}
\end{figure}
