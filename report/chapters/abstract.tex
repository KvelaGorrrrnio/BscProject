\begin{abstract}
  This thesis describes the implementation of two \textit{reversible flowchart languages}, RL and SRL, along with the inversion and translation of these. Two interfaces --- a command-line interface and a web-based user interface --- will be designed and implemented as well. Reversible flowcharts model reversible (imperative) programs similar to how conventional flowcharts model conventional programs; reversible flowcharts, however, can be inverted and thus perform backward program execution. Reversible flowchart languages are the product of concretising the theory behind these; two such proposed languages are RL and SRL, which resemble a low-level assembler language and a high-level structured language, respectively.

We will, for each of the two languages, go through the definition of their respective interpreter and, based on this, argue for our concrete implementation. Any potential changes will be defined and explained thoroughly.

Furthermore, we will describe our implementation of two types of program transformations. The first of these is, not surprisingly, program inversion; we would like to be able to invert arbitrary programs written in one any of the two languages.

The other is program translation; translating structured SRL to unstructured RL and, as proven to be possible by the Structured Program Theorem for reversible languages, translation from unstructured RL to structured SRL.\

We will then go through the implementation of a web-based interface for the interpreters and their program transformations. This could potentially be put on a domain and work as an educational tool in the future.

Finally, we will write some RL- and SRL programs that serve as tests and demonstrations.
\end{abstract}

% skriv i nutid, passivform
