%%%%%%%%%%%%%%%%%%%%%%%%%%%%%%%%%%
% Package for making LaTeX properly handle utf8 characters set and danish language rules
\usepackage[utf8]{inputenc}
\usepackage[english]{babel}
\usepackage{float}


%%%%%%%%%%%%%%%%%%%%%%%%%%%%%%%%%%
% Package for changing to a nicer font
\usepackage [T1]{fontenc}

%%%%%%%%%%%%%%%%%%%%%%%%%%%%%%%%%%
% Package for conctroling the text area
\usepackage[margin=2.5cm]{geometry}

%%%%%%%%%%%%%%%%%%%%%%%%%%%%%%%%%%
% Package for inserting clickable hyperlinks in pdf versions as produced by pdflatex
\usepackage[colorlinks = true,
            linkcolor = black,
            urlcolor  = blue,
            citecolor = blue,
            anchorcolor = blue]{hyperref}
%%%%%%%%%%%%%%%%%%%%%%%%%%%%%%%%%%
% Package for including figures. TeX and thus LaTeX was developped before the existence of directory file-structures, but the graphicspath let's you add directories, that the \includegraphics will search.
\usepackage{graphicx}
\graphicspath{{figures/}}

%%%%%%%%%%%%%%%%%%%%%%%%%%%%%%%%%%
% Package for typesetting programs. Listings does not support fsharp, but a little modification goes a long way
\usepackage{listings}
\usepackage{color,xcolor}

%%%%%%%%%%%%%%%%%%%%%%%%%%%%%%%%%%
% Package for extended math settings, e.g. \eqref
\usepackage{amsmath}
\usepackage{amssymb}


%%%%%%%%%%%%%%%%%%%%%%%%%%%%%%%%%%
% Listings styling
\definecolor{greybg}{RGB}{246,247,248}
\definecolor{codeblue}{RGB}{15,126,227}
\definecolor{codegreen}{rgb}{0,0.6,0}
\definecolor{codegray}{rgb}{0.5,0.5,0.5}
\definecolor{codepurple}{rgb}{0.58,0,0.82}
\definecolor{backcolour}{rgb}{0.95,0.95,0.92}
\definecolor{codegrey}{rgb}{0.4,0.4,0.4}

\lstdefinestyle{mystyle}{
    backgroundcolor=\color{greybg},
    commentstyle=\color{codegreen},
    keywordstyle=\color{codeblue},
    numberstyle=\tiny\color{codegray},
    stringstyle=\color{codepurple},
    basicstyle=\footnotesize\ttfamily,
    breakatwhitespace=false,
    breaklines=true,
    captionpos=b,
    keepspaces=true,
    numbers=left,
    numbersep=5pt,
    showspaces=false,
    showstringspaces=false,
    showtabs=false,
    tabsize=2
}

\lstset{style=mystyle, lineskip=1pt}
\lstset{language=C,
  basicstyle=\fontsize{9}{9}\ttfamily,
  keywordstyle=\color{blue}\ttfamily,
  stringstyle=\color{red}\ttfamily,
  commentstyle=\color{gray}\ttfamily,
  morecomment=[l][\color{magenta}]{\#}
}

\lstdefinestyle{DOS}
{
    backgroundcolor=\color{black},
    basicstyle=\footnotesize\color{white}\ttfamily,
    numbers=none,
    framextopmargin=6pt,
    framexbottommargin=6pt,
    frame=tb, framerule=0pt,
    stringstyle=\color{white}
}

%% Own math shortcuts
\newcommand{\EE}[1]{\mathbb{E}\left(#1\right)}
\newcommand{\PP}[1]{\mathbb{P}\{#1\}}
\newcommand{\PPP}[2]{\mathbb{P}_{#1}\{#2\}}
\newcommand{\pmtx}[1]{\begin{pmatrix}#1\end{pmatrix}}
\newcommand{\given}{\ \vert\ }
\newcommand{\CP}{\xrightarrow[]{\ \mathbb{P}\ } }
\newcommand{\CD}{\xrightarrow[]{\ \mathcal{L}\ } }
\newcommand{\kudos}{\textbf{\texttt{KUDOS}} }
\newcommand{\type}[1]{\textcolor{codepurple}{\texttt{#1}}}
\newcommand{\const}[1]{\textcolor{codeblue}{\texttt{#1}}}
\newcommand{\var}[1]{\textcolor{codegrey}{\texttt{#1}}}
\newcommand{\cmd}[1]{\colorbox{black}{\texttt{\textcolor{white}{\$\ #1}}}}

\usepackage{mathtools}
\DeclarePairedDelimiter\ceil{\lceil}{\rceil}
\DeclarePairedDelimiter\floor{\lfloor}{\rfloor}

%% Remove indentation
%\setlength\parindent{0pt}

%% Tables
\usepackage{booktabs}
\newcommand{\tblcell}[2][l]{%
  \begin{tabular}[#1]{@{}l@{}}#2\end{tabular}}

% Tikz
\usepackage{tikz}
\usetikzlibrary{arrows.meta}

% Frontpage
\usepackage{epstopdf} % Package for including eps images
\usepackage{eso-pic} % AddToShipoutPicture
